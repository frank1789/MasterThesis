\chapter{Hardware}
\lettrine[lines=3]{T}{his} chapter describe hardware used in the project, in particular thermal camera modile
and interface with other hardware and how to setup parts ad well the whole system
to be able recevie video frames from camera module.

\section{Thermal Camera module Lepton 2.5}

The thermal camera module we use in thi project, we will go through its specification, modes of capture and communication protocol for both video frame transfer and camera control.
Information about the camera are extracted from official documentation.
% TODO CHECK DOCUMENTATION AND LINK IN REFERENCE-BIBLIOGRAPHI

% TODO INSERT IMAGE FLIR LEPTON

\subsection{Specification}
In this project we use a Lepton 2.5 thermal camera module, made by the company FLIR which is currently one of the top manufactures of thermal camera solutions. It contains a sensor sensitive to long wave infra red (LWIR) light in range from 8 to 14 \si{\micro\meter}.
The camera module is smaller than a dime and provides decent images with its 160 by 120 pixels resolution. The effective frame rate of the camera is only 8.8 \si{\hertz}, however for our needs this is not a problem. The camera only requires low voltage supply and has small power consumption of around 140 \si{\milli\watt}. See table 3.1 for more specifications.
For better manipulation with the camera module we use a breakout board (figure 3.1) with a housing for the Lepton camera module. The breakout board provides better physical accessibility, improves heat dissipation and increases input voltage supply range to 3-5 volts, as it has its own regulated power supply. This power supply provides the camera module with three necessary voltages: 1.2, 2.8 and 2.8-3.1 \si{\volt}. The breakout board also supplies the camera with master clock signal. [14] The camera uses two interfaces for communication:
\begin{itemize}
    \item SPI for transferring video frames from the camera to a SPI master device.
    \item I$^{2}$C for receiving control commands from the I$^{2}$C master device.
\end{itemize}
Even though the name of the project include the keyword low-cost, we need to think of this statement with respect to the thermal camera market. The Lepton 3 thermal camera module can be considered low-cost when compared to other thermal camera devices available—as it costs around $250$ (2018) 4.
This could however be considerably more when compared to other nonthermal solutions, however way more than if we would utilize a simple infrared counting sensors for example.
Spectral range Array format Pixel size Thermal sensitivity FOV horizontal FOV diagonal Depth of field Lens type Output format Clock speed Input voltage Power dissipation Dimensions
% 8 to 14 𝜇𝑚
% 160 × 120 pixels
% 12 𝜇𝑚
% <50 𝑚𝐾
% 56∘
% 71∘
% 28 𝑐𝑚 to ∞
% f/1.1 silicon doublet
% 14-bit Y14 raw flux or 24-bit RGB888 false color 25 𝑀𝐻𝑧
% 2.8 𝑉, 1.2 𝑉, 2.8-3.1 𝑉
% 140 𝑚𝑊 operating, 5 𝑚𝑊 shutdown mode 11.8 × 12.7 × 7.2 𝑚𝑚

\begin{table}[htb]
    \centering
    \label{tab:thcamspecifications}
    \caption{Key Specifications}
    \begin{tabular}{l c c}
        \hline
                                    &   FLIR Lepton 2.5 &          \\
        \hline
        \rowcolor{aliceblue!85} Resolution (h x v)	        &   80 $\times$ 60  &   pixels  \\
        Spectral Range	                                    &   8  to 14        &   \si{\micro\meter}   \\
        \rowcolor{aliceblue!85} Horizontal Field of View	&   51              &   \degree \\
        Thermal Sensitivity	        &   < 50                &   \si{\milli\kelvin}  \\
        \rowcolor{aliceblue!85} Frame Rate	                &   < 9             &   \si{\hertz} \\
        Control Interface	                                &   I$^{2}$C        &               \\
        \rowcolor{aliceblue!85} Video Interface	            &   SPI             &               \\
        Promised Time to Image	                            &   < 0.5           &   \si{\second}    \\
        \rowcolor{aliceblue!85} Integral Shutter		    &   yes             &   \\
        Radiometry	                                        &   14-bit pixel value  &           \\
        \rowcolor{aliceblue!85} Operating Power             &	$\sim$150       &   \si{\milli\watt} \\
        \hline
\end{tabular}
\end{table}
