\chapter{Future work}
\label{chap:future-work}
%
% preface chapter
\lettrine[lines=3]{G}{iven}the results achieved, further improvements to the 
software are possible to the entire project structure. 
It will be possible to modify the neural model to make it more efficient and 
precise by using new backbone networks that perform the classification with the
possibility of extracting new ones features from the pictures. 
The dataset can be reviewed, expanded or calibrated for more specific
objectives, I provide images rich in detail and if needed I can provide possible
images made with dedicated hardware. 
This project has focused attention on implementation and training of a neural
network model for embededd systems, therefore on an architecture that would give
the best results. 
Although a 3D dataset has been used for area shooting, in the future for a
dataset made with images captured in real life flight conditions, this would
lead to more detail and realism on the perception of the environment. 
Then by improving the data provided to the process training it will be possible
to obtain a more accurate response of the model. 
In fact, please note that for the inference on the thermal images a dataset was
made by FLIR with a different hardware from the one used in this project. 
Making a new dataset with the use of the Lepton 2.5 camera will allow you to
capture images with a different resolution which allow you to capture
different characteristics for the benefit of the results of the image analysis
both in the training phase and in response to the computer vision problem. 
If on the one hand the results of the neural network are connected to the
dataset to train them, on the other hand you can intervene by changing the
hyper-parameters of the model. 
By modifying the hyper-parameters it is possible to carry out more selective
calibrations during training by changing the balance of accuracy and speed.
Although the compression offered by TensorFLow Lite is an excellent tool to work
on embedded devices, the quantization shows a drift of the results with respect
to the uncompressed model. Starting from the version of TensorFLow 2.0 it is
possible to train the network with quantized values. 
This will allow you to obtain better results during the execution of the
inference and a drastic reduction of false positives.