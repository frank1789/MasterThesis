\chapter{Future work}
\label{chap:future-work}
%
% preface chapter
\lettrine[lines=3]{T}{his}

Visti i riusltati raggiunti sono possibili ancora miglioramenti al software ma,
all'intera struttura del progetto. Sarà possibile modificare il modello neurale
per renderlo più efficiente e preciso ricorrendo a nuovi rete backbone che
effettuano la classificazione o la possibilità di estrarre nuove
caratteristiche.
Il dataset può essere rivisto, ampliato o tarato per obiettivi più specifici,
fornedo immagini ricche di dettaglio ed eventualmete realizzate con hardware
dedicato. 
Questo progetto ha focalizzato l'attenzione sulla realizzazione e addestramento
di un modello di rete neurale per sistemi embededd, quindi sull'archiettura che
fornisse i risultati migliori. Nonostente si sia utilzzato un dataset realizzato
in 3D per le riprese aree, in futuro un dataset realizzato in vere condizioni di
volo potrebe portare dei benefici.
Si ricorda infatti che per l'inferenza sulle immagini termiche si è usato un
dataset realizzato dalla FLIR, quindi realizzado uno con hardware specifico può
migliorare le prestazioni della rete neurale. Se da una parte molti dei
risultati delle rete neurale sono collegate al dataset utilizzato per
addestrarle, altrettando si può fare modificando i parametri della rete.
Modificando i valori degli iper-parametri è possibile operare delle tarature più
selettive per durante l'addastramento del modello. Sebbene la compressione
offerta da TensorFLow Lite sia un ottimo strumento per portare su dispotivi
embedded mobili ed IoT, la quantizzazione mostra una deviazione dei risulati
seppur piccola rispetto il modello non compresso.
D'altra parte a partire dalla versione di TensorFLow 2.0 è disponibile
l'addestrmaneto quantizzato, cioè durante la fase di addestramento della modello
neurale, è possibile tenere conto dell'effetto della quantizzazione. Questo
potrebbe garantire migliori risulati durante l'inferenza è una drastica
riduzione di falsi positivi.
