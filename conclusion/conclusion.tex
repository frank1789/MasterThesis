\chapter{Conclusion}
\label{chap:conclusion}
%
% preface chapter
\lettrine[lines=3]{T}{he aim} of this thesis is to increase automation and 
calculation skills on board the drone with the help of the machine learnig, 
solving the computer vision problem and in particular object detection problem.
The system uses a standard camera with a resolution of 8 MegaPixels for acquire
images that will be processed by the neural network, as it can rely on the
Lepton 2.5 thermal camera to detect objects that they emit heat. In particular,
the software created is designed to exploit at most the parallelization of the
processor, precisely to manage the flows videos from the cameras. 
The Raspberry Pi 3b has wifi connection therefore the software uses it on a TCP
socket to send two flows to a possible device connected to the same network.
Thus it is possible to carry out the inference with the model directly on board
neural that in the case of color images will try to identify the carpet landing,
while in the case of the thermal camera the second model neural will try to
identify any obstacles, in this case people, to avoid them during the execution
of the maneuvers.
As discussed the use of deep learning allows to solve a specific problem
calibrating the response on the input data to have the classification in
response of the object and its position within the frame captured by the camera.
In particular using the single shot detection model based on mobilenet, it
provides excellent results both in the classification of subjects framed both
the position of bounding box that frames the same.
The fine tuning techniques thus allow to train the in a short time model to
classify a specific target. Clearly this as it is said it is not possible to do
it on board the drone, but high devices are required computational services such
as the University HPC cluster made available.
Using the TensorFlow Lite framework to compress the model once it has been
trained, so you can choose to use floating point or integer values. 
In particular, using a compressed model allows it to be used on embedded
devices, typically less performing, as they can be computer board card, such as
the Raspberry Pi 3b used in this project, thus providing acceptable performance.
Use instead of processors dedicated as TPU mounted on the Coral Dev-Board shows
how it still is possible to increase the response speed of the neural model
thanks the heavy use of 8-bit integers. 
Although the benefits are significant reached the compression penalizes the
final result at the end of the inference process since the big difference
between the architectures of the processors, the optimizations introduced by the
compiler and the user affect the recognition of the target thus producing a
slight increase in cases of false positives.
Although it is possible to code any model in TensorFlow Lite this is it is not
always possible, in fact using custom layers they can prevent the compression of
the trained model.
For this reason, during the development phase of the neural network, it was
decided to use a network from the TensorFlow API unlike the preliminary phases
in which a customized model had been adopted, therefore the model is more
compatible with TensorFlow Lite. 
Is presents the right balance between accuracy and speed of execution, this also
ensured full compatibility with the TPU.
Ultimately it is possible to remark that the work done here produces a system
able to increase the capabilities of a drone thanks to the help of deep learning
without resorting to pre-existing solutions deriving from closed licenses or
commercial combined with devices for sale.