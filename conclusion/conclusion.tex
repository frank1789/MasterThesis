\chapter{Conclusion}
\label{chap:conclusion}
%
% preface chapter
\lettrine[lines=3]{T}{his}

L'obiettivo di questa tesi è aumentare le capacità di automazione e calcolo a
bordo con l'ausilio del machine learnig, in particolar modo risolvendo il
problema comaputer vision ed in particolar modo di object detection.
Il sitema utilizza una telecamera standard con risoluzione da 8 MegaPixels per
acquisire immagini che verrano processate dalla rete neurale, cosi come pone può
fare affidamento sulle tlecamera termica Lepton 2.5 per rilevare oggetti che
emettono calore. In particolare il software realizzato è concepito per sfuttare
al massimo la parallelizzazione del processore, appunto per gestire i flussi
video provienti dalla telecamere. Dispone di una connessione TCP per inviare i
due flussi ad un eventuale dispotivo connesso alla stessa rete.
Così è possibile effettuare direttamente a bordo l'inferenza con il modello
neurale che nel caso di immagini a colori cercherà di indentificare il tappetto
di atterraggio, mentre nel caso della telecamera termica il secondo modello
neurale cercherà di indentificare eventuali ostacoli, in questo caso persone,
per evitarle durante l'esecuzione delle manovre. 
Come discusso l'uso del deep learning permette di risolvere un problema specifico
taradno la risposta su i dati in input per avere in risposta la classificazione
dell'oggetto e la sua posizione del all'interno del frame catturato dalla
fotocamera. In particolare utilizzando il modello di single shot detection
basato su mobilenet, fornisce ottimi risulati sia nella classificazione degi
soggetti inquadrati sia la posizione di bounding box che incornicia il medesimo.
Le tecniche di fine tuning permettono cosi di addestrare in breve tempo il
modello per classificare uno specifico target. Chiaramente questo come è detto
non è possibile farlo a bordo del drone, ma sono necessari dipositivi ad elevate
prestazioni computazioni come il cluster HPC di Ateneo messo a disposizione.
L'uso del framework TensorFlow Lite comprimere il modello una volta addesstrato,
per cui è possibile scegliere su usare valori in virgola mobile o interi. In
particolare modo l'utillizzare un modello compresso permette di essere
utilizzato su dispotivi embedded, tipicamente meno performanti, quali possono
essere computer board card, come ad esempio la Raspberry Pi 3b usata in questo
progetto, fornendo quindi prestazioni accettabili. L'uso invece di processori
dedicati come TPU montato sulla Coral Dev-Board dimostra come ancora sia
possibile aumentare la velocità di risposta del modello neurale grazie
all'utilizzo spinto degli interi a 8-bit. Sebbene siano notevoli i benefici
raggiunti la compressione penalizza il risultato finale a conclusione del
processo inferenza in quanto la grande differenza tra le architetture dei
processori, le ottimizzazioni intordotte dal compilatore e dall'utente
influiscono sul riconoscimento del target producendo così un leggero aumento dei
casi di falsi positivi.
Malgrado sia possibile codificare qualsiasi modello in TensorFlow Lite questo
non è sempre posiibile, infatti utilizzando layer custom possono impedire la
compressione del modello addestrato.
Per questo motivo si è optato ad utilizzare non un modello personalizzato come
all'inizio, ma un modello maggiormente compatibile con TensorFlow Lite e che
presentasse il giusto bilanciamento tra accuratezza e velocità di esecuzione,
questo ha anche garantito la piena compatibilità anche la TPU.
In definitiva è possibile rimarcare che il lavoro qui svolto produce un sistema
in grado di aumentare le capacità di un drone grazie l'ausilio del deep learning
senza ricorrere a soluzione preesistenti derivanti da licenze chiuse o
commerciali abbiante a dispositivi in vendita.