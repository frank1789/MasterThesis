\chapter*{Abstract}
\label{chap:abstract}
%
%
\lettrine[lines=3]{T}{he aim} of this thesis is to increase the capabilities of
an unmanned aerial vehicles (UAVs) using convolutional neural networks for
classification and object detection from images captured by the camera.
In recent times, the use of UAVs has increased both in the military and civilian
fields, such as for example in traffic control, surveillance, deliveries,
photography and exploration.
Within the UAV family, helicopters are preferred to fixed-wing aircraft
especially for their vertical take-off and landing capacity, and also for their
manoeuvrability.
These are used in activities where environments and circumstances are dangerous
to humans.
In parallel successes of deep learning techniques in solving many complex tasks
such as planning, localization, control and perception by starting from sensor
data acquired in real environments make it suitable in autonomous robotic
applications.\\
The development of dedicated hardware and software such as CUDA for Nvidia GPUs
which shows best performances in computer vision speech recognition signal signal
processing applications and so on.\\
A further aspect taken into consideration is the continuous evolution in the
field of Information Technology (IT) driven by the continual request for
decreasing capacity and costs.
So embedded systems represent a now important slice of the entire IT sector, in
particular they guide thanks to their mutual dependence on hardware and software
in the mobile field and the Internet of Things (IoT). These devices take
advantage of application characteristics to optimize the design.
Thus the introduction of platforms dedicated to neural computation has pushed
the adoption of specific energy efficient and powerful processors.\\

\noindent This project focuses the activity of providing the drone with greater 
awareness of the surrounding area by allowing the resolution of Computer Vision
problems through the Convolution Neural Network.
The approach used through the use of Convolution Neural Network algorithms
solves two of the problems: one for classification and one for object dection,
returning in the first case the label of the subject, in the second it provides
the location of the target within the image acquired.\\
Although software solutions already exist and are imminent in commercial drones
as closed license applications. The need to provide an open source solution is a
necessary choice for the whole sector.\\
On the other hand, the use of open source software requires that new solutions
that are modifications, upgrades or new implementations, require that they also
be released under the same license. 
n addition, specific architectures, such as ARM and TPU, were used to give a
solution to the problem. Such architectures in partiocolarment and optimized aid
promise high performance with reduced consumption during processing.



\ref{chap:introduction}\\


\noindent Chapter \ref{chap:hardware} introduces the hardware used to make the
prototype by analyzing the computer boards in particular the Raspberry Pi 3b and
the Google Coral Dev-Board.\\
These although similar differ for the mounted processor if for the first board
we find an ARM cortex-A53 32-bit processor for the second board we find a
processor of the same class but with 64-bit processing flanked by a Tensor
Processor Unit for neural calculus.\\
There are also cameras used to acquire images, in particular there is the
Raspberry PiCamera V2 equipped with sensor 8-Mega-pixels. 
Instead, for the acquisition of thermal images, a FLIR Lepton 2.5 with a sensor
capable of providing images of the size $ 80 \times 60 $ was preferred. 
Finally, it is presented in the logic that makes the use of the Tensor Processor
Unit so attractive and its implications in energy consumption and tensor
calculation.\\

\noindent In chapter \ref{chap:software} the software developed to be executed
on the boards is analyzed. 
In particular, they justify the choices to use the Qt framework in
order to guarantee ease during the development phase and cross-platform support.
The critical functions within the program are analyzed in detail, highlighting
the design and implementation choices.\\ 
As well as a comparison of the performances between the tested architectures, in
particular: Intel i7 (x86\_64), ARM cortex-A53 (armv7l) and ARM cortex-A53
(aarch64).\\


\noindent The design and implementation of the dataset is presented in chapter
\ref{chap:neuralnetworks}, this represents a fundamental aspect in the training
process of the neural network.
In detail, the organization of the dataset is discussed and how it should be
implemented in order to guarantee the solution of the object detection problem.
In particular, the annotation work carried out on the images and how they are
then processed by the neural network.
The choice of the neural network to be trained is discussed in depth by carrying
out a trade-off between performances, that is, having the shortest execution
time to effect the inference. 
Correctness in the suggested answer to the classification and object detection
problem. 
Without forgetting the possibility of being performed on specific hardware.\\


\ref{chap:conclusion}\\



\noindent The last chapter \ref{chap:future-work} presents possible future
developments, strategies that will benefit the various aspects examined in the
thesis.
In particular, diversify and characterize the dataset in order to make the
classification and object detection even more characteristic and specific.
Adopting different and more optimized neural network models. 
As well as different training techniques and tools.\\ 
Lastly, a more efficient and efficient optimization of the code in order to
limit and reduce time and consumption of resources and energy for the benefit of
greater execution speed and downtime with implications also on the energy
aspect.