\chapter{Software}
\label{chap:software}
\lettrine[lines=3]{I}{n} this chapter we will discuss the software used to
control the cameras presented in the chapter. This software was designed to have
a minimum number of dependencies and was specifically optimized to exploit the
hardware already presented.

\chapter{Intoduction}
\label{chap:introduction}
%
% preface chapter
\lettrine[lines=3]{T}{his}
%
the software created is based on the paradigm of object-oriented programming. It
is a question of keeping the dependence on other libraries and frameworks to a
minimum, trying to guarantee the portability of the source code on platforms
with different hardware. The language chosen is C ++ with an intuitive use of
the Qt framework thus once again guaranteed portability, as well as the
graphical interface and high data performance the feature of compiled languages.
The software thus created inside presents the graphic structures of Qwidget, the
libraries to manage the Raspberry Pi camera seen in chapter 4 and the FLIR
Lepton SDK.

Once started, the application runs the main screen that can be divided into
three main areas: the preview section, the fusion section, the sidebar of the
commands. In the first section two widgets are shown that show the video streams
captured by the two cameras in separate threads, this thanks to the extensive
use of the ``\textbf{signal \& slot}"\footnote{Signals and slots are used for
communication between objects. The signals and slots mechanism is a central
feature of Qt and probably the part that differs most from the features provided
by other frameworks. Signals and slots are made possible by Qt's meta-object
system.\cite{qtsignalslot}} made available by Qt. In the sidebar of the
controls, there are the operations that allow you to change the color palette of
the thermal image, capture an image, select a particular blending method for the
two video streams and see them combined in the central section. The central
section, as anticipated, shows the superimposition of the thermal map on the RGB
image captured by the traditional photo camera.


